\chapter{Basic Architecture}

\section{Addressing}

The basic addressable unit is the 8-bit byte. 

\section{Data Type}

\subsection{Unsigned Integer}

\subsubsection{Byte}

\begin{bytefield}{8}
    \bitheader[endianness=big]{7,0} \\
    \bitbox{8}{}
\end{bytefield}

\subsubsection{Halfword}

\begin{bytefield}{16}
    \bitheader[endianness=big]{15,0} \\
    \bitbox{16}{}
\end{bytefield}

\subsubsection{Word}

\begin{bytefield}{32}
    \bitheader[endianness=big]{31,0} \\
    \bitbox{32}{}
\end{bytefield}


\subsection{Signed Integer}

\subsubsection{Signed byte}
\begin{bytefield}[bitheight=\widthof{~Sign~}]{8}
    \bitheader[endianness=big]{7,6,0} \\
    \bitbox{1}{\rotatebox{90}{Sign}} 
    \bitbox{7}{}
\end{bytefield}

\subsubsection{Signed halfword}
\begin{bytefield}[bitheight=\widthof{~Sign~}]{16}
    \bitheader[endianness=big]{15,14,0} \\
    \bitbox{1}{\rotatebox{90}{Sign}} 
    \bitbox{15}{}
\end{bytefield}

\subsubsection{Signed word}
\begin{bytefield}[bitheight=\widthof{~Sign~}]{32}
    \bitheader[endianness=big]{31,30,0} \\
    \bitbox{1}{\rotatebox{90}{Sign}} 
    \bitbox{31}{}
\end{bytefield}

\section{CPU Registers}

An Athena processor includes two types of registers: general purpose registers (GPR) and control/status registers (CRS). 

\subsection{General Purpose Registers}

\begin{tabular}{ |p{3cm}|p{3cm}|p{4cm}|  }
    \hline
    \multicolumn{3}{|c|}{General Purpose Registers} \\
    \hline
    Name& Alternate name & Comment \\
    \hline
    r0  & zero & Hardwired zero\\
    r1  & a0   & Func arg 0 \\
    r2  & a1   & Func arg 1 \\
    r3  & a2   & Func arg 2 \\
    r4  & a3   & Func arg 3 \\
    r5  & a4   & Func arg 4 \\
    r6  & a5   & Func arg 5 \\
    r7  & a6   & Func arg 6 \\
    r8  & a7   & Func arg 7 \\
    r9  & v0   & Return value \\
    r10 & v1   & Return value \\
    r11 & s0   & Saved register \\
    r12 & s1   & Saved register  \\
    r13 & s2   & Saved register  \\
    r14 & s3   & Saved register  \\
    r15 & s4   & Saved register  \\
    r16 & s5   & Saved register  \\
    r17 & s6   & Saved register  \\
    r18 & t0   & Temporary register \\
    r19 & t1   & Temporary register \\
    r20 & t2   & Temporary register \\
    r21 & t3   & Temporary register \\
    r22 & t4   & Temporary register \\
    r23 & t5   & Temporary register \\
    r24 & t6   & Temporary register\\
    r25 & k0   & Reserved for kernel \\
    r26 & k1   & Reserved for kernel \\
    r27 & at   & Reserved for Assembler \\
    r28 & sp   & Stack Pointer \\
    r29 & gp   & Global Poiter \\
    r30 & fp   & Frame Pointer \\
    r31 & ra   & Return address \\
    
    \hline
\end{tabular}

\subsection{Control and Status Registers}

\begin{tabular}{ |p{3cm}|p{3cm}|p{1cm}|p{1cm}|p{6cm}| }
    \hline
    \multicolumn{5}{|c|}{Special Registers} \\
    \hline
    N° & name & read & write & Description \\
    \hline
    0 & isa & \ok & \no & Machine ISA Register \\
    1 & status & \ok & \ok & Status Register \\
    2 & cause & \ok & \no & Cause Register \\
    3 & trapvec & \no & \ok & Trap Vector Base Address Regiter\\
    4 & epc & \no & \no & Exeption Program Counter \\
    5 & pc & \no & \no & Program Counter \\
    \hline
\end{tabular}


\subsubsection{Machine ISA Register}

\begin{bytefield}[bitwidth=\widthof{IE}]{32}
    \bitheader[endianness=big]{31,1,0} \\
    \bitbox{30}{-}
    \bitbox{1}{F}
\end{bytefield}

\begin{itemize}
    \item F: Single-Precision Floating-Point Extension
    \item C: Cryptography Extension
\end{itemize}

\subsubsection{Status Register}

\begin{bytefield}[bitwidth=\widthof{ET}]{32}
    \bitheader[endianness=big]{31,3,2,1,0} \\
    \bitbox{4}{impl}
    \bitbox{4}{vers}
    \bitbox{20}{-}
    \bitbox{1}{ET}
    \bitbox{1}{P} 
    \bitbox{1}{S}
\end{bytefield}

\begin{itemize}
    \item S: supervisor mode (1: supervisor, 0: user mode)
    \item P: paging (1: enabled, 0: disabled)
    \item ET: enable traps (1: enabled, 0: disabled) 
\end{itemize}

\subsubsection{Trap Vector Base Address Regiter}

\begin{bytefield}{32}
    \bitheader[endianness=big]{31,1,0} \\
    \bitbox{31}{base}
    \bitbox{1}{M}
\end{bytefield}

\subsubsection{EPC}

\subsubsection{Program Counter}
